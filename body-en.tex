\section{Introduction}

The purpose of this paper is to present the current status of the
system presented in \cite{kroger08:rameau}, Rameau. The system was
first designed for automatic harmonic analysis, but it has been
extended with basic support for computational musicology as well.

\section{The system}
\label{sec:system}

Rameau has two basic interfaces; the command line interface gives the
user access to all options and functionality and the web interface is
just a basic front-end where the user can either choose a file to be
analyzed, or type the notes of a four-part chorale.

Rameau has nine algorithms for chord finding and four algorithms for
roman numeral analysis. Figures \ref{fig:chord-name-analysis} and
\ref{fig:roman-analysis} show the analysis result for the chord
finding and roman numeral analysis, respectively. Each row shows the
result for one algorithm and the last row show the expected answer, as
predicted \nota{usar verbo melhor} in the answer sheet. The answer
sheet is as simple ascii file with a co \nota{completar}

The complete answer sheet for the chord name analysis is:

\begin{verbatim}
Em D/F# G B7/F# Em B/D# C/E D7/F# [A] Em7
Am7/C Am7 D G G G/B D D Am/C Em/B [A] B 
(B7 [a]) Em
\end{verbatim}

and the complete answer sheet for the roman numeral analysis is:

\begin{verbatim}
e: i V6/III III V4.3 i V6 VI6 V6.5/III
III i G: ii6.5 ii7 V I I I6 V -
e: iv6 i6.4 - V7 - i
\end{verbatim}


\begin{figure}
  \centering
  
  \caption{Chord name analysis}
  \label{fig:chord-name-analysis}
\end{figure}
\begin{figure}
  \centering
  
  \caption{Roman numeral analysis}
  \label{fig:roman-analysis}
\end{figure}

\section{Algorithms}
\label{sec:algorithms}

\nota{falar sobre algoritmos (duh!)}

\section{Computational musicology}
\label{sec:comp-music}

\section{Visualization}
\label{sec:visualization}

\section{Conclusion}
\label{sec:conclusion}



%%% Local Variables: 
%%% mode: latex
%%% TeX-master: "icmc2009"
%%% End: 
